
\documentclass[a4paper]{moderncv}
\moderncvtheme[blue]{classic} % or classic, casual, oldstyle
\usepackage[T2A,T1]{fontenc} % T2A for Cyrillic, T1 for Western European
\usepackage[utf8]{inputenc}
\usepackage[scale=0.86]{geometry}
\usepackage{comment} % For multi-line comments if needed


% Inline bibliography using filecontents
\begin{filecontents}[overwrite]{cv_publications.bib}
@inbook{AIAA,
  author = {Nishikawa, Hiroaki and Nakashima, Yoshitaka and Mikushin, Dmitry and Lee, Jeff},
  title = {A Reduced-Memory Multicolor Gauss-Seidel Relaxation Scheme for Implicit Unstructured-Polyhedral-Grid CFD Solver on GPU},
  year = {2025},
  booktitle = {AIAA AVIATION 2025 Forum},
  note = {(to appear)}
}

@inproceedings{mikushin2014kernelgen,
  author = {Mikushin, Dmitry and Likhogrud, Nikolay and Zhang, Eddy Z and Bergström, Christopher},
  title = {KernelGen--The Design and Implementation of a Next Generation Compiler Platform for Accelerating Numerical Models on GPUs},
  year = {2014},
  booktitle = {2014 IEEE International Parallel \& Distributed Processing Symposium Workshops},
  pages = {1011--1020},
  organization = {IEEE}
}

@inproceedings{kuzmin2017end,
  author = {Kuzmin, Andrey and Mikushin, Dmitry and Lempitsky, Victor},
  title = {End-to-end learning of cost-volume aggregation for real-time dense stereo},
  year = {2017},
  booktitle = {2017 IEEE 27th International Workshop on Machine Learning for Signal Processing (MLSP)},
  pages = {1--6},
  organization = {IEEE}
}

@article{brumm2015scalable,
  author = {Brumm, Johannes and Mikushin, Dmitry and Scheidegger, Simon and Schenk, Olaf},
  title = {Scalable high-dimensional dynamic stochastic economic modeling},
  year = {2015},
  journal = {Journal of Computational Science},
  volume = {11},
  pages = {12--25},
  publisher = {Elsevier}
}

@inproceedings{scheidegger2018rethinking,
  author = {Scheidegger, Simon and Mikushin, Dmitry and Kubler, Felix and Schenk, Olaf},
  title = {Rethinking large-scale economic modeling for efficiency: Optimizations for gpu and xeon phi clusters},
  year = {2018},
  booktitle = {2018 IEEE International Parallel and Distributed Processing Symposium (IPDPS)},
  pages = {610--619},
  organization = {IEEE}
}

@misc{mikushin2012kernelgen,
  author = {Mikushin, Dmitry and Likhogrud, Nicolas},
  title = {KernelGen--a toolchain for automatic GPU-centric applications porting},
  year = {2012}
}

@inproceedings{georgieva2023falkor,
  author = {Georgieva Belorgey, Mariya and Dandjee, Sofia and Gama, Nicolas and Jetchev, Dimitar and Mikushin, Dmitry},
  title = {Falkor: Federated Learning Secure Aggregation Powered by AESCTR GPU Implementation},
  year = {2023},
  booktitle = {Proceedings of the 11th Workshop on Encrypted Computing \& Applied Homomorphic Cryptography},
  pages = {11--22}
}

@article{jian2020gpu,
  author = {Jian, Nianchuan and Mikushin, Dmitry and Yan, Jianguo and Barriot, Jean-Pierre and Wu, Yajun and Wang, Guangli},
  title = {A GPU-based phase tracking method for planetary radio science applications},
  year = {2020},
  journal = {Measurement Science and Technology},
  volume = {31},
  number = {4},
  pages = {045902},
  publisher = {IOP Publishing}
}

@inproceedings{nguyen2021hiq,
  author = {Nguyen, Damien and Mikushin, Dmitry and Man-Hong, Yung},
  title = {HiQ-ProjectQ: Towards user-friendly and high-performance quantum computing on GPUs},
  year = {2021},
  booktitle = {2021 Design, Automation \& Test in Europe Conference \& Exhibition (DATE)},
  pages = {1056--1061},
  organization = {IEEE}
}

@article{stepanenko2008numerical,
  author = {Stepanenko, Victor and Mikushin, Dmitry},
  title = {Numerical modeling of mezoscale dynamics in the atmosphere and tracer transport above hydrologically inhomogeneous land},
  year = {2008},
  journal = {Computational Technologies},
  volume = {13},
  number = {Special issue 3},
  pages = {104--110}
}

@inproceedings{mikushin2010implementation,
  author = {Mikushin, Dmitry and Stepanenko, Victor},
  title = {The implementation of regional atmospheric model numerical algorithms for Cell Broadband Engine Architecture-based clusters},
  year = {2010},
  booktitle = {Parallel Processing and Applied Mathematics: 8th International Conference, PPAM 2009, Wroclaw, Poland, September 13-16, 2009. Revised Selected Papers, Part I 8},
  pages = {525--534},
  organization = {Springer Berlin Heidelberg}
}

@poster{ecem2019,
  author = {Mikushin, Dmitry and Korotaev, Kirill and MacInnes, William Joseph},
  title = {On coupling of EyeStalker algorithm with USB3.0 camera for affordable eye tracking},
  year = {2019},
  booktitle = {20th European Conference on Eye Movements},
  pages = {392}
}


\end{filecontents}

% BibLaTeX for bibliography
\usepackage[backend=bibtex, style=numeric, sorting=ydnt, maxnames=999, minnames=999]{biblatex}
% style=authoryear-comp, bibstyle=authoryear, citestyle=authoryear-comp are other options
% sorting=nyt (name, year, title), ydnt (year (desc), name, title)
\addbibresource{cv_publications.bib} % Bib file name

% Define bib environment for moderncv compatibility
\defbibenvironment{bibliography}
{\list
  {\printtext[labelnumberwidth]{% label format from numeric.bbx
        \printfield{labelprefix}%
        \printfield{labelnumber}}}
  {\setlength{\labelwidth}{\hintscolumnwidth}%
    \advance\leftmargin\labelsep}%
  \sloppy\clubpenalty4000\widowpenalty4000}
{\endlist}
{\item}
\renewcommand*{\bibnamedash}{\mbox{\textemdash\space}}% for repeated authors
% Fix spacing between volume and number fields
\DeclareFieldFormat[article]{volume}{#1}
\DeclareFieldFormat[article]{number}{#1}
\renewbibmacro*{volume+number+eid}{%
  \printfield{volume}%
  \setunit*{.\space}% Add space after period between volume and number
  \printfield{number}%
  \setunit{\addcomma\space}%
  \printfield{eid}}

\AtBeginDocument{\recomputelengths} % Recalculate lengths (important)

% Personal Information
\firstname{Dmitry}
\familyname{Mikushin, PhD} % moderncv doesn't have a separate title command like this, it's part of familyname

\address{1815 Clarens, Switzerland}{}{}
\mobile{+41 78 925 90 90}
\email{dmitry@kernelgen.org}
\homepage{mikush.in}
\social[github]{dmikushin}
\social[linkedin]{dmikushin}
\photo[64pt]{photo.jpg}

\AfterPreamble{\hypersetup{colorlinks,urlcolor=blue}} % Make links blue

\begin{document}
\maketitle

\textbf{Work authorization in Switzerland}: work permit B

\section{PROFESSIONAL SUMMARY}

I work at the intersection of research and industrial programming (C/C++, Fortran, Python), where classical supercomputing (MPI, OpenMP) meets modern hardware (GPUs, TPUs, embedded ASICs) and performance optimization.

As co--founder of Purple Gaze Inc., I've developed eyetracking systems that are both more affordable and faster than many existing solutions.

I've earned my PhD from the University of Lausanne, where I developed specialized supercomputing software for economics and finance applications.

My expertise in CUDA, OpenCL, OpenACC, and Machine Learning has established me as an effective consultant in these fields.

I provide effective GPU--accelerated solutions to organizations facing complex computational challenges. My skills help teams successfully port legacy algorithms to utilize NVIDIA H100/GH200 GPU architectures, implement memory--optimized algorithms for better compute--memory balance, and develop portable cross--platform solutions that work reliably across different computing environments.

As a regular user of AI--assisted development tools, I incorporate LLMs into my daily workflow to improve productivity and solve technical challenges more efficiently.


\section{KEY ACHIEVEMENTS}

\begin{itemize}
\item \textbf{GPU--Accelerated Algorithms Design}: Created and implemented numerous high--performance GPU algorithms in many domains: CFD, economics, computer vision, homomorphic encryption, self--driving simulators, and even radio astronomy
\item \textbf{Cross--Platform Development}: Successfully ported multiple research and production software packages to diverse operating systems, architectures, and platforms
\item \textbf{Research Impact}: Co--authored significant research papers and industrial presentations in high--performance computing and computer vision
\item \textbf{Entrepreneurial Success}: Co--founded Purple Gaze Inc. and spearheaded development of the company's first ``Foxy'' eyetracking device
\end{itemize}


\section{CORE COMPETENCIES}

\begin{itemize}
\item \textbf{Technical Team Lead/CTO}: Proven ability to develop technical strategy for long--term projects, lead by example through technical excellence, and effectively direct mixed human/AI teams
\end{itemize}

\begin{itemize}
\item \textbf{HPC/Engineering}: Mastery of the complete development \& support cycle for HPC applications on Linux clusters: programming, multi--architecture parallelization, debugging, and performance optimization. Experienced with large--scale codebases including MSC Nastran, PyTorch, and MIOpen.
\end{itemize}

\begin{itemize}
\item \textbf{GPU Acceleration Expert}: Specialized in memory footprint optimization for large--scale simulations, legacy solver migration to NVIDIA H100/GH200 GPUs, and implementation of highly optimized algorithms for superior compute--memory balance.
\end{itemize}

\begin{itemize}
\item \textbf{Research \& Development}: Skilled at exploring new technologies and transferring that knowledge to others, designing comprehensive experiments to analyze hardware/software performance, and translating findings into practical methods and tools.
\end{itemize}

\begin{itemize}
\item \textbf{AI--Enhanced Development}: Daily practitioner of LLM--assisted coding workflows, leveraging AI tools (shell\_gpt, VSCode Agent) to streamline development, troubleshoot complex code, and overcome technical obstacles.
\end{itemize}

\begin{itemize}
\item \textbf{Embedded SW/HW}: Developer of specialized embedded machine vision systems. Designed and implemented Linux firmware and real--time processing services for resource--constrained ARM processors (eyetracking), with expertise in real--time device--host communication.
\end{itemize}

\begin{itemize}
\item \textbf{Compilers Development}: Deep understanding of compiler internals with contributions to LLVM. Creator of KernelGen -- an innovative auto--parallelizing Fortran/C compiler for NVIDIA GPUs.
\end{itemize}


\section{TECHNICAL SKILLS}

\begin{itemize}
\item \textbf{Programming Languages}: C/C++, CUDA/HIP/OpenCL, Fortran, Python, Perl, Bash; an active learner of Rust
\item \textbf{Development Tools}: CMake/make, git, gdb, vim, tmux
\item \textbf{AI Tools}: shell\_gpt, ollama, gguf, VSCode agent
\item \textbf{High Performance Computing}: GPU acceleration, OpenMP (multicore and GPU offload), MPI, algorithm optimization
\item \textbf{GPU Expertise}: NVIDIA H100/GH200, Parallel Programming, Memory optimization, Performance tuning, Kepler/Volta assembler
\item \textbf{CFD \& Numerical Methods}: Unstructured solvers, Gauss--Seidel methods, Fréchet derivatives, Memory reduction techniques
\item \textbf{Compiler Development}: LLVM contributions, Clang/LLVM plugins developer
\item \textbf{Embedded Systems}: UART, U--boot, WiringPi, ADBD, Allwinner, Rockchip, ATTiny85, Cypress FX3, uvc/genicam camera vision
\item \textbf{Cloud Technologies}: Docker, Singularity, ssh
\item \textbf{Operating Systems}: Linux, FreeBSD
\item \textbf{Hardware}: Electronics prototyping, soldering, circuit checking, milling machine operation
\end{itemize}


\section{PROFESSIONAL EXPERIENCE}
\cventry{2019--present}{Co--Founder \& CTO}{Purple Gaze Inc.}{Lausanne Area, Switzerland}{}{\footnotesize \begin{itemize}
\item Developed research--quality high--performance eyetracking hardware and software stack
\item Created company's first 'Foxy' eyetracking device product
\item Lead development of high--precision embedded systems for eyetracking and machine vision
\item Direct technical strategy and engineering team in creating cutting--edge vision technologies
\end{itemize}}
\vskip 2pt
\cventry{2023--2025}{Senior Software Engineer}{Hexagon Manufacturing Intelligence}{Renens, Vaud, Switzerland (Remote)}{}{\footnotesize \begin{itemize}
\item Implemented GPU support for Cradle CFD and MSC Nastran simulation software on NVIDIA platforms
\item Led development of GPU--accelerated unstructured CFD solver that achieved 15\% performance improvement on a single NVIDIA H100 GPU compared to 384 CPU threads on 4× AMD EPYC 7763 CPUs
\item Developed reduced--memory multicolor Gauss--Seidel method using Fréchet derivative, achieving 45\% memory reduction
\item Implemented GH200 support by porting code to ARM64 with Clang
\item Created portable OpenMP code for both multicore and GPU offload scenarios
\item Presented research and achievements at NVIDIA GPU Technology Conference (GTC) 2025
\end{itemize}}
\vskip 2pt
\cventry{2022--2023}{HIP/C++ Developer (Contractor)}{AMD}{Zug, Switzerland (Remote)}{}{\footnotesize \begin{itemize}
\item Developed GPU optimizations for MIOpen machine learning engine
\item Enhanced performance of machine learning frameworks through advanced C++ and CUDA implementations
\item Collaborated with cross--functional teams to improve AMD's GPU computing capabilities
\end{itemize}}
\vskip 2pt
\cventry{2019--2023}{Research Assistant}{University of Lausanne (UNIL)}{Lausanne Area, Switzerland}{}{\footnotesize \begin{itemize}
\item Conducted research in GPU computing applications
\item Collaborated with academic teams on parallel computing projects
\end{itemize}}
\vskip 2pt
\cventry{2018--2019}{CUDA/C++ Developer}{Valeo}{Remote}{}{\footnotesize \begin{itemize}
\item Designed and implemented efficient GPU kernels for self--driving car lidar simulation (ADAS)
\item Optimized performance for real--time processing requirements
\item Contributed to autonomous vehicle technology development
\end{itemize}}
\vskip 2pt
\cventry{2017--2018}{CUDA/C++ Developer}{Excellian}{Remote}{}{\footnotesize \begin{itemize}
\item Implemented GPGPU for high--performance Monte--Carlo backpricing valuation for financial customers
\item Completely rewrote Scala code into C++ \& CUDA with skip--ahead optimizations for Sobol QRNG
\item Prepared backend for integration with Murex financial platform
\end{itemize}}
\vskip 2pt
\cventry{2015--2019}{Assistant}{University of Zurich}{Zürich Area, Switzerland}{}{\footnotesize \begin{itemize}
\item Applied GPGPU techniques for computational economics in Dr. Simon Scheidegger's research team
\item Contributed to academic research bridging high--performance computing and economic modeling
\item Supported teaching and research activities in parallel computing
\end{itemize}}
\vskip 2pt
\cventry{2014--present}{Owner / CUDA Educator \& Researcher}{Applied Parallel Computing LLC}{Montreux, Vaud, Switzerland}{}{\footnotesize \begin{itemize}
\item Created original courses in GPU computing, CUDA, OpenACC and related technologies
\item Organized and led a small team of consultants for EMEA region
\item Provided specialized training in NVIDIA and AMD GPU computing technologies to German automotive industry clients and universities
\item Consulted customers in industry and academia on--site
\item Expertise in CUDA and HIP frameworks for high--performance computing
\end{itemize}}
\vskip 2pt
\cventry{2012--2016}{Doctoral Assistant}{USI Università della Svizzera italiana}{Lugano, Switzerland}{}{\footnotesize \begin{itemize}
\item Taught Master--level courses in computer science
\item Conducted doctoral research in parallel computing
\item Supported academic programs through teaching and research assistance
\end{itemize}}
\vskip 2pt
\cventry{2013--2013}{Visiting Scholar}{Rutgers University}{New Brunswick, New Jersey}{}{\footnotesize \begin{itemize}
\item Assisted research activities of Prof. Eddy Zheng Zhang
\item Contributed to academic research in parallel computing and GPU applications
\item Collaborated with international research teams on advanced computing projects
\end{itemize}}
\vskip 2pt
\cventry{2011--2012}{Technical Lead}{Applied Parallel Computing LLC}{Dubna, Moscow Region}{}{\footnotesize \begin{itemize}
\item Led company's GPGPU R\&D projects and training services
\item Directed technical teams in developing parallel computing solutions
\item Managed client relationships for technical training programs
\end{itemize}}
\vskip 2pt
\cventry{2009--2011}{Developer Technology Engineer}{NVIDIA}{Moscow, Russian Federation}{}{\footnotesize \begin{itemize}
\item Developed GPU spectral solver benchmark and GPU kernels generator for COSMO model (Deutscher Wetterdienst)
\item Implemented SPU--interacting radix sort for rigid bodies broad phase algorithm on Cell Broadband Engine processor (Sony PlayStation 3)
\item Created proof--of--concept Multi--GPU applications to demonstrate hardware benefits to customers
\item Contributed to PhysX engine Linux port and developed experimental Tegra/ARM ports
\item Provided CUDA/HPC customer support and training sessions
\end{itemize}}
\vskip 2pt
\section{EDUCATION}
\cventry{2019--2023}{Doctor of Philosophy (PhD) in Business Analytics}{University of Lausanne (UNIL)}{Lausanne, Switzerland}{}{Dissertation: High--performance computing approaches to solve large--scale dynamic models in economics and finance\\ Focus areas: Dynamic programming, Compilers, C++, Combinatorics, GPGPU, High Performance Computing (HPC)}
\vskip 2pt
\cventry{2003--2008}{MSc in Computational Mathematics and System Programming}{Lomonosov Moscow State University (MSU)}{Moscow, Russia}{}{Department: Department of Computational Technologies and Modelling (Institute of Numerical Mathematics, Russian Academy of Science)}
\vskip 2pt
\section{PUBLICATIONS}
% Publications are managed via bibliography
% Use \nocite{*} to include all references, or \nocite{key1,key2,...} for specific ones
\nocite{AIAA,mikushin2014kernelgen,kuzmin2017end,brumm2015scalable,scheidegger2018rethinking,mikushin2012kernelgen,georgieva2023falkor,jian2020gpu,nguyen2021hiq,stepanenko2008numerical,mikushin2010implementation,ecem2019}
 % This will contain \section{...} \cventry{...} etc.

% Print bibliography if there are publications
\printbibliography[heading=none]

\end{document}
