%% start of file `template_en.tex'.
%% Copyright 2006-1008 Xavier Danaux (xdanaux@gmail.com).
%
% This work may be distributed and/or modified under the
% conditions of the LaTeX Project Public License version 1.3c,
% available at http://www.latex-project.org/lppl/.
\documentclass[a4paper]{moderncv}
% moderncv themes
\moderncvtheme[blue]{classic} % optional argument are 'blue' (default), 'orange', 'red', 'green', 'grey' and 'roman' (for roman fonts, instead of sans serif fonts)
%\moderncvtheme[green]{classic} % idem
% character encoding
\usepackage[T2A]{fontenc}
\usepackage[utf8]{inputenc} % replace by the encoding you are using
% adjust the page margins
\usepackage[scale=0.86]{geometry}
%\setlength{\hintscolumnwidth}{3cm} % if you want to change the width of the column with the dates
%\AtBeginDocument{\setlength{\maketitlenamewidth}{6cm}} % only for the classic theme, if you want to change the width of your name placeholder (to leave more space for your address details
\AtBeginDocument{\recomputelengths} % required when changes are made to page layout lengths
\firstname{Dmitry}
\familyname{Mikushin}
\address{via Pelloni 2}{6900 Lugano}
\mobile{+4178-925-9090}
%\mobile{+19083420880}
\email{dmitry@parallel-computing.pro}
\extrainfo{skype: maemarcus}
\photo[72pt]{picture.jpg}

\AfterPreamble{\hypersetup{colorlinks,urlcolor=blue}}

\begin{document}
\maketitle

\section{Academic experience}
\cventry{2015--present}{Research Associate}{University of Zurich}{Institut f\"{u}r Banking und Finance}{Switzerland}{\footnotesize HPC/GPU programming \& optimization for high-dimensional dynamic stochastic economic models.}
\vskip 2pt
\cventry{2012--present}{Doctoral Assistant}{University of Lugano}{Switzerland}{}{\footnotesize Assisting education and research activities lead by Prof. Olaf Schenk.}
\vskip 2pt
\cventry{2013}{Visiting Scholar}{Rutgers University}{New Jersey, US}{}{\footnotesize Joint research activities with Prof. Eddy Zheng Zhang. Analyzed the efficiency of atomics on different families of NVIDIA GPUs. Tuned GPU kernels optimizations in KernelGen compiler.}
\vskip 2pt
\cventry{2011--2012}{Book editor}{}{}{}{\footnotesize Parallel computing on GPU. Architecture and CUDA programming model (in Russian). Alexey Boreskov, Alexander Kharlamov, Nikolay Markovskiy, Dmitry Mikushin, Evgeny Mortikov, Alexander Myltsev, Nikolai Sakharnykh, Vladimir Frolov. MSU Publishing (June 2012).}
\vskip 2pt
\cventry{2009--2010}{Lecturer}{GPU Programming Course \& HPC Summer School}{Lomonosov Moscow State University}{}{\footnotesize Presented lectures and hands-ons on integrating CUDA with MPI and multi-threading, and on debugging CUDA applications.}
\vskip 2pt
\cventry{2008--2011}{Junior scientist}{Supercomputer simulation laboratory for climate modeling, Research Computing Center}{Lomonosov Moscow State University, Russia}{}{{\footnotesize Numerical and performance evaluation of various mesoscale and regional models. Experimented with deploying key model dynamics blocks on CellBE and GPUs.}}
\vskip 2pt
\cventry{2006--2007}{Contractor}{Global Energy Problems Lab}{Moscow Energy Institute}{}{\footnotesize Implemented toolbox for Voronoi tesselation and regression analysis in C\#.}

\section{Professional experience}
\cventry{2014--present}{Owner}{Applied Parallel Computing LLC (CUDA Education \& Research in EMEA)}{}{\url{http://parallel-computing.pro/}}{}
\cventry{2011--present}{Project lead}{{KernelGen} open-source compiler toolchain}{\url{http://kernelgen.org/}}{}{\footnotesize Design and development of LLVM-based compiler for identifying parallel loops in C/Fortran code and converting them into GPU kernels. Strategic planning, interacting with community, creating new partnerships.}{}
\vskip 2pt
\cventry{2011--2012}{CTO}{Applied Parallel Computing LLC (CUDA Education \& Research in EMEA)}{Dubna, Moscow Region, Russia}{\url{http://parallel-computing.pro/}}{\footnotesize Managing technological aspects in company's GPGPU training and software development business. Created course list on comprehensive CUDA training program, implemented original presentations and hands-ons, later used in CUDA 4.x Handbook in Russian. Served as trainer on events in Germany, Ireland and Russia. Organizing and reviewing work of 7 company's contracted trainers/developers. Responsible for interaction with customers and partners worldwide.}
\vskip 2pt
\cventry{2009--2011}{DevTech Engineer}{NVIDIA}{Moscow, Russia}{}{\footnotesize Ported parts of numerical weather prediction models onto GPUs: spectral solver benchmark (Russian Met Office), GPU kernels generator for COSMO model (Deutscher Wetterdienst et al). Supported customers and developers on CUDA programming in HPC applications, provided training sessions. PhysX game physics engine: implemented SPU-interacting radix sort for rigid bodies broad phase algorithm on Cell Broadband Engine processor (Sony PlayStation 3), made first experimental Tegra/ARM ports of PhysX engine, helped with Linux port.}

\section{Awards}
\cventry{2013}{PhD fellowship}{Rutgers University, Department of Computer Science}{}{}{}
\vskip 2pt
\cventry{2011}{CUDA Certificate 016-2011/29.10.2011}{NVIDIA}{Moscow}{Massively parallel processors, CUDA architecture and programming environment}{}
\vskip 2pt
\cventry{2008}{PhD fellowship}{Institute of Numerical Mathematics, Russian Academy of Science}{}{}{}
\vskip 2pt
\cventry{2008}{T-Platforms PowerXCell 8i Programmers Contest, second award}{Optimization of mathematical modeling package for hydrodynamics ``GeoPhyCell''}{}{}{}
\vskip 2pt
\cventry{2008}{Best Student Diploma, second award}{Numerical modeling of mesoscale aerosol transfer due to hydrological inhomogeneity of the boundary layer}{}{}{} 

\section{Education}
\cventry{2012--present}{PhD}{University of Lugano}{Institute of Computational Science}{Switzerland}{}{}
\vskip 2pt
\cventry{2008--2011}{PhD (ABD -- passed qualification and comprehensive examinations)}{Institute of Numerical Mathematics, Russian Academy of Science}{Moscow}{}{}
\vskip 2pt
\cventry{2003--2008}{Specialist (5-year B.S. + M.S program)}{Faculty of Computational Mathematics and Cybernetics}{Lomonosov Moscow State University}{\textit{Computational Technologies and Modeling}}{}

\section{Master thesis}
\cvline{title}{\emph{Numerical modeling of mesoscale aerosol transfer due to hydrological inhomogeneity of the boundary layer}}
\cvline{supervisors}{Dr. Vasily N. Lykossov, Dr. Victor M. Stepanenko}
\cvline{}{\footnotesize Implemented and analyzed Smolarkiewicz transport scheme, the positive-definite method of Lax–Wendroff class. Resulting source code was incorporated into regional non-hydrostatic model of atmosphere and boundary layer (NH3D) and used to trace passive aerosol. Experiments with real terrains showed significant numerical accuracy improvement both in mass conservation and approximation order over leapfrog and first order transport schemes.}

% Publications from a BibTeX file
\nocite{*}
\bibliographystyle{plainyr-rev}
\bibliography{publications}

\section{Selected talks}
{\small
\cventry{}{Dmitry Mikushin, Nikolay Likhogrud, Sergey Kovylov. KernelGen: A Prototype of Auto-parallelizing Fortran/C compiler for NVIDIA GPUs}{GPU Technology Conference 2013}{\href{http://registration.gputechconf.com/quicklink/fEoVwIQ}{available online}}{}{}}

\section{Active skills}
\cventry{CS/Research}{\small Explore new environments/software and teach others to use them, design \& perform experiments to analyse hardware/software properties and generalize findings into practically useful methods/tools}{}{}{}{}
\cventry{HPC/Engineer}{\small Fluency in full development \& support cycle of HPC applications for Linux clusters: programming, parallelization for different architectures, debugging, profiling}{}{}{}{}
\cventry{Compilers Dev}{\small Basic knowledge of compilers internal structure, contributions to LLVM. Designed and developed {KernelGen} -- a prototype of auto-parallelizing Fortran/C compiler for NVIDIA GPUs, targeting numerical modelling code}{}{}{}{}
\cventry{GPU low-level}{\small Experience with NVIDIA GPU binary format and Fermi/Kepler assembler, profiling \& optimization}{}{}{}{}
\cventry{Numericals}{\small Practical experience with linear solvers, PDEs and related cache-aware optimizations}{}{}{}{}
\cventry{NWP}{\small Engineer-level experience with numerical weather prediction models: WRF-ARW, COSMO}{}{}{}{}

\section{Teaching}
\cventry{2014}{Parallel \& Distributed Computing Lab}{University of Lugano}{}{}{5 practical assignments on code porting, profiling \& optimization for TBB, MPI, CUDA, CellBE and Xeon Phi.}{}
\cventry{2014}{Parallel \& Distributed Computing}{University of Lugano}{}{}{The fundamentals of concurrent execution. Threading in Java. The basics of OpenMP and MPI.}{}
\cventry{2013}{Parallel \& Distributed Computing Lab}{University of Lugano}{}{}{Configuration \& deployment of scientific codes on modern GPU-enabled HPC facilities, by example of SWE tsunami simulation model and CSCS ``T\"odi'' cluster}{}

\section{Languages}
\cvlanguage{English}{\small fluent technical}{}
\cvlanguage{Russian}{\small native}{}

\end{document}
