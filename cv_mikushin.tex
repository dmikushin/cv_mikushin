%% start of file `template_en.tex'.
%% Copyright 2006-1008 Xavier Danaux (xdanaux@gmail.com).
%
% This work may be distributed and/or modified under the
% conditions of the LaTeX Project Public License version 1.3c,
% available at http://www.latex-project.org/lppl/.
\documentclass[a4paper]{moderncv}
% moderncv themes
\moderncvtheme[blue]{classic} % optional argument are 'blue' (default), 'orange', 'red', 'green', 'grey' and 'roman' (for roman fonts, instead of sans serif fonts)
%\moderncvtheme[green]{classic} % idem
% character encoding
\usepackage[T2A]{fontenc}
\usepackage[utf8]{inputenc} % replace by the encoding you are using
% adjust the page margins
\usepackage[scale=0.86]{geometry}
%\setlength{\hintscolumnwidth}{3cm} % if you want to change the width of the column with the dates
%\AtBeginDocument{\setlength{\maketitlenamewidth}{6cm}} % only for the classic theme, if you want to change the width of your name placeholder (to leave more space for your address details
\AtBeginDocument{\recomputelengths} % required when changes are made to page layout lengths
\firstname{Dmitry}
\familyname{Mikushin}
%\address{via Pelloni 2}{6900 Lugano}
%\mobile{+4178-925-9090}
%\mobile{+19083420880}
\address{Moscow Region}{142290 Puschino}
\mobile{+79260570677}
\email{dmitry@parallel-computing.pro}
%\email{dmikushin@luxoft.com}
\extrainfo{skype: maemarcus}
\photo[72pt]{photo}

\AfterPreamble{\hypersetup{colorlinks,urlcolor=blue}}

\begin{document}
\maketitle

\section{Academic experience}
\cventry{2015--present}{Research Associate}{University of Zurich}{Institut f\"{u}r Banking und Finance}{Switzerland}{\footnotesize 
Group of Dr. Simon Scheidegger. Computing global solutions to annually calibrated dynamic stochastic general equilibrium models for policy analysis (overlapping generation models or OLG). Given prototypes from economists within the group, developing production code for large hybrid computing systems equipped with NVIDIA or Intel accelerators (CSCS/Piz Daint, NERSC/Cori). Fine-tuning of value function interpolation kernels on a sparse grid with linear or polynomial basis (AVX, CUDA, Intel Thread Building Blocks). Co-authored a journal paper.}
\vskip 2pt
\cventry{2012--2015}{Doctoral Assistant}{University of Lugano}{Switzerland}{}{\footnotesize Group of Prof. Olaf Schenk. Assisting education and research activities. Worked on manual optimization and designed tools for automatic optimization of stencil codes in seismic and weather prediiction models (CUDA, OpenACC). Main author for a conference paper.}
\vskip 2pt
\cventry{2013}{Visiting Scholar}{Rutgers University}{New Jersey, US}{}{\footnotesize Group of Prof. Eddy Zheng Zhang. Analyzed the efficiency of atomics on different families of NVIDIA GPUs. Tuned GPU kernels optimizations in KernelGen compiler.}
\vskip 2pt
\cventry{2008--2011}{Junior scientist}{Supercomputer simulation laboratory for climate modeling, Research Computing Center}{Lomonosov Moscow State University, Russia}{}{{\footnotesize Numerical and performance evaluation of various mesoscale and regional models. Experimented with porting key model dynamics blocks on Cell Broadband Engine and GPU architectures.}}
\vskip 2pt
\cventry{2006--2007}{Contractor}{Global Energy Problems Lab}{Moscow Energy Institute}{}{\footnotesize Implemented toolbox for Voronoi tesselation and regression analysis in C\#.}

\section{Industrial experience}
\cventry{2018--present}{System Analyst (part time)}{REG.RU}{}{Moscow}{\footnotesize 
Overseeing the development of Cloud GPU services for scientific and industrial machine learning. Helping with internal GPU-based spin-off products.}
\cventry{2017--present}{Lead Technical Expert}{Luxoft Inc.}{}{Moscow}{\footnotesize 
GPGPU for high-performance Monte-Carlo backpricing valuation (financial customer). Complete rewrite of Scala code into C++ \& CUDA, skip-ahead optimizations for Sobol RNG. Design and implementation of efficient GPU kernels for lidar simulation (automotive customer).}
\cventry{2014--present}{Owner}{Applied Parallel Computing LLC (CUDA Education \& Research in EMEA)}{}{\url{http://parallel-computing.pro/}}{}
\cventry{2011--present}{Project lead}{{KernelGen} open-source compiler toolchain}{\url{http://kernelgen.org/}}{}{\footnotesize Design and development of LLVM-based compiler for identifying parallel loops in C/Fortran code and converting them into GPU kernels. Strategic planning, interacting with community, creating new partnerships.}{}
\vskip 2pt
\cventry{2011--2012}{CTO}{Applied Parallel Computing LLC (CUDA Education \& Research in EMEA)}{Dubna, Moscow Region, Russia}{\url{http://parallel-computing.pro/}}{\footnotesize Managing technological aspects in company's GPGPU training and software development business. Created course list on comprehensive CUDA training program, implemented original presentations and hands-ons, later used in CUDA 4.x Handbook in Russian. Served as trainer on events in Germany, Ireland and Russia. Organizing and reviewing work of 7 company's contracted trainers/developers. Responsible for interaction with customers and partners worldwide.}
\vskip 2pt
\cventry{2009--2011}{DevTech Engineer}{NVIDIA}{Moscow, Russia}{}{\footnotesize Ported parts of numerical weather prediction models onto GPUs: spectral solver benchmark (Russian Met Office), GPU kernels generator for COSMO model (Deutscher Wetterdienst et al). Supported customers and developers on CUDA programming in HPC applications, provided training sessions. PhysX game physics engine: implemented SPU-interacting radix sort for rigid bodies broad phase algorithm on Cell Broadband Engine processor (Sony PlayStation 3), made first experimental Tegra/ARM ports of PhysX engine, helped with Linux port.}

\section{Awards}
\cventry{2013}{PhD fellowship}{Rutgers University, Department of Computer Science}{}{}{}
\vskip 2pt
\cventry{2011}{CUDA Certificate 016-2011/29.10.2011}{NVIDIA}{Moscow}{Massively parallel processors, CUDA architecture and programming environment}{}
\vskip 2pt
\cventry{2008}{PhD fellowship}{Institute of Numerical Mathematics, Russian Academy of Science}{}{}{}
\vskip 2pt
\cventry{2008}{T-Platforms PowerXCell 8i Programmers Contest, second award}{Optimization of mathematical modeling package for hydrodynamics ``GeoPhyCell''}{}{}{}
\vskip 2pt
\cventry{2008}{Best Student Diploma, second award}{Numerical modeling of mesoscale aerosol transfer due to hydrological inhomogeneity of the boundary layer}{}{}{} 

\section{Education}
\cventry{2012--2015}{PhD studies}{University of Lugano}{Institute of Computational Science}{Switzerland}{}{}
\vskip 2pt
\cventry{2008--2011}{PhD (ABD -- passed qualification and comprehensive examinations)}{Institute of Numerical Mathematics, Russian Academy of Science}{Moscow}{}{}
\vskip 2pt
\cventry{2003--2008}{Specialist (5-year B.S. + M.S program)}{Faculty of Computational Mathematics and Cybernetics}{Lomonosov Moscow State University}{\textit{Computational Technologies and Modeling}}{}

\section{Master thesis}
\cvline{title}{\emph{Numerical modeling of mesoscale aerosol transfer due to hydrological inhomogeneity of the boundary layer}}
\cvline{supervisors}{Dr. Vasily N. Lykossov, Dr. Victor M. Stepanenko}
\cvline{}{\footnotesize Implemented and analyzed Smolarkiewicz transport scheme, the positive-definite method of Lax–Wendroff class. Resulting source code was incorporated into regional non-hydrostatic model of atmosphere and boundary layer (NH3D) and used to trace passive aerosol. Experiments with real terrains showed significant numerical accuracy improvement both in mass conservation and approximation order over leapfrog and first order transport schemes.}

% Publications from a BibTeX file
\begin{thebibliography}{1}

\bibitem{IPDPS} Simon Scheidegger, Dmitry Mikushin, Felix Kubler, and Olaf Schenk.
\newblock Rethinking large-scale economic modeling for efficiency: optimizations for GPU and Xeon Phi clusters.
\newblock {\em IEEE IPDPS}, pp. 610-619, 2018.

\bibitem{brumm2015scalable}
Johannes Brumm, Dmitry Mikushin, Simon Scheidegger, and Olaf Schenk.
\newblock Scalable high-dimensional dynamic stochastic economic modeling.
\newblock {\em Journal of Computational Science}, 11:12--25, 2015.

\bibitem{Mikushin:2014:KDI:2672598.2672916}
Dmitry Mikushin, Nikolay Likhogrud, Eddy~Z. Zhang, and Christopher
  Bergstr\"{o}m.
\newblock Kernelgen -- the design and implementation of a next generation
  compiler platform for accelerating numerical models on gpus.
\newblock In {\em Proceedings of the 2014 IEEE International Parallel \&
  Distributed Processing Symposium Workshops}, IPDPSW '14, pages 1011--1020,
  Washington, DC, USA, 2014. IEEE Computer Society.

\bibitem{PPAM}
Dmitry Mikushin and Victor Stepanenko.
\newblock The implementation of regional atmospheric model numerical algorithms
  for {Cell} {Broadband} {Engine} {Architecture} -based clusters.
\newblock In Roman Wyrzykowski, Jack Dongarra, Konrad Karczewski, and Jerzy
  Wasniewski, editors, {\em PPAM (1)}, volume 6067 of {\em Lecture Notes in
  Computer Science}, pages 525--534. Springer, 2009.

\end{thebibliography}

\section{Selected talks}
{\small
\cventry{}{Dmitry Mikushin, Nikolay Likhogrud, Sergey Kovylov. KernelGen: A Prototype of Auto-parallelizing Fortran/C compiler for NVIDIA GPUs}{GPU Technology Conference 2013}{\href{http://registration.gputechconf.com/quicklink/fEoVwIQ}{available online}}{}{}}

\section{Active skills}
\cventry{CS/Research}{\small Explore new environments/software and teach others to use them, design \& perform experiments to analyse hardware/software properties and generalize findings into practically useful methods/tools}{}{}{}{}
\cventry{HPC/Engineer}{\small Fluency in full development \& support cycle of HPC applications for Linux clusters: programming, parallelization for different architectures, debugging, profiling}{}{}{}{}
\cventry{Bitcoin mining}{\small Basic understanding of SHA256 and Zero-knowledge proof. Optimization of Equihash miner for NVIDIA GPUs. Available on \href{https://github.com/dmikushin/nheqminer}{GitHub}}{}{}{}{}
\cventry{Compilers Dev}{\small Basic knowledge of compilers internal structure, contributions to LLVM. Designed and developed {KernelGen} -- a prototype of auto-parallelizing Fortran/C compiler for NVIDIA GPUs, targeting numerical modelling code}{}{}{}{}
\cventry{GPU low-level}{\small Experience with NVIDIA GPU binary format and Fermi/Kepler assembler, profiling \& optimization}{}{}{}{}
\cventry{Numericals}{\small Practical experience with linear solvers, PDEs and related cache-aware optimizations}{}{}{}{}
\cventry{NWP}{\small Engineer-level experience with numerical weather prediction models: WRF-ARW, COSMO}{}{}{}{}

\section{Teaching}
\cventry{2014}{Parallel \& Distributed Computing Lab}{University of Lugano}{}{}{5 practical assignments on code porting, profiling \& optimization for TBB, MPI, CUDA, CellBE and Xeon Phi.}{}
\cventry{2014}{Parallel \& Distributed Computing}{University of Lugano}{}{}{The fundamentals of concurrent execution. Threading in Java. The basics of OpenMP and MPI.}{}
\cventry{2013}{Parallel \& Distributed Computing Lab}{University of Lugano}{}{}{Configuration \& deployment of scientific codes on modern GPU-enabled HPC facilities, by example of SWE tsunami simulation model and CSCS ``T\"odi'' cluster}{}

\section{Languages}
\cvlanguage{English}{\small fluent technical}{}
\cvlanguage{Russian}{\small native}{}

\end{document}
